%=========================================================================================================================================
%=========================================================================================================================================
\chapter{Implementierungen}
%=========================================================================================================================================
%=========================================================================================================================================

\begin{lstlisting}[language=Java, caption=Implementierung des GET CHALLENGE Kommandos,label=lst:appendix:impl:challenge1]
package at.ac.uniklu.simulator.commands.iso7816;

import java.util.Random;

import at.ac.uniklu.simulator.commands.CommandHandler;
import at.ac.uniklu.simulator.commands.CommandPattern;
import at.ac.uniklu.simulator.commands.ISO7816DefaultReturnCodes;
import at.ac.uniklu.simulator.commands.RAPDU;
import at.ac.uniklu.simulator.util.Enum1;
import at.ac.uniklu.simulator.util.HEXByteString;

public class GetChallengeHandler extends CommandHandler
{
  public String getIdentifier()
  {
    return "GetChallenge";
  }

  public CommandPattern getCommandPattern()
  {
    return new CommandPattern("XX", "84");
  }

  protected Enum1 returnDataStatus()
  {
    return EnumStatus.needed;
  }

  public Enum1 dataStatus()
  {
    return EnumStatus.notallowed;
  }

  public RAPDU preExecute()
  {
    RAPDU ret = super.preExecute();
    if (ret != null)
      return ret;

    if (command.getP1().getByte() != 0 || command.getP2().getByte() != 0)
      return ISO7816DefaultReturnCodes.get(ISO7816DefaultReturnCodes.codes.SW_WRONG_PARAMETERS);
    return null;
  }


  public RAPDU execute()
  {
    Random r = new Random();
    int randnum = command.getNe();

    byte[] rbytes = new byte[randnum];
    r.nextBytes(rbytes);

    RAPDU ret = ISO7816DefaultReturnCodes.get(ISO7816DefaultReturnCodes.codes.SW_NO_ERROR);
    ret.setData(new HEXByteString(rbytes));

    this.getOpSys().getCurrentSession().setChallenge(new HEXByteString(rbytes));

    return ret;
  }
}
\end{lstlisting}

% Der Quellcode \ref{lst:appendix:impl:challenge2} enth�lt den Code der \inlinejavacode{GetChallengeHandler}-Klasse ohne
% Kommentare, aber mit platzsparendem Einr�cken. Somit konnte ein ung�nstiger Seitenumbruch vermieden werden.
%
% \begin{lstlisting}[language=Java, caption=Implementierung des GET CHALLENGE Kommandos,label=lst:appendix:impl:challenge2]
% package at.ac.uniklu.simulator.commands.iso7816;
%
% import java.util.Random;
%
% import at.ac.uniklu.simulator.commands.CommandHandler;
% import at.ac.uniklu.simulator.commands.CommandPattern;
% import at.ac.uniklu.simulator.commands.ISO7816DefaultReturnCodes;
% import at.ac.uniklu.simulator.commands.RAPDU;
% import at.ac.uniklu.simulator.util.Enum1;
% import at.ac.uniklu.simulator.util.HEXByteString;
%
% public class GetChallengeHandler extends CommandHandler
% {
%   public String getIdentifier() {
%     return "GetChallenge";
%   }
%
%   public CommandPattern getCommandPattern() {
%     return new CommandPattern("XX", "84");
%   }
%
%   protected Enum1 returnDataStatus() {
%     return EnumStatus.needed;
%   }
%
%   public Enum1 dataStatus() {
%     return EnumStatus.notallowed;
%   }
%
%   public RAPDU preExecute() {
%     RAPDU ret = super.preExecute();
%     if (ret != null) return ret;
%     if (command.getP1().getByte() != 0 || command.getP2().getByte() != 0)
%       return ISO7816DefaultReturnCodes.get(ISO7816DefaultReturnCodes.codes.SW_WRONG_PARAMETERS);
%     return null;
%   }
%
%   public RAPDU execute() {
%     Random r = new Random();
%     int randnum = command.getNe();
%
%     byte[] rbytes = new byte[randnum];
%     r.nextBytes(rbytes);
%
%     RAPDU ret = ISO7816DefaultReturnCodes.get(ISO7816DefaultReturnCodes.codes.SW_NO_ERROR);
%     ret.setData(new HEXByteString(rbytes));
%
%     this.getOpSys().getCurrentSession().setChallenge(new HEXByteString(rbytes));
%
%     return ret;
%   }
% }
% \end{lstlisting}

%-----------------------------------------------------------------------------------------------------------------------------------------
\section{Kein A.1 ohne A.2}
%-----------------------------------------------------------------------------------------------------------------------------------------

\cleardoublepage
%=========================================================================================================================================
%=========================================================================================================================================
% EOF
%=========================================================================================================================================
%=========================================================================================================================================
